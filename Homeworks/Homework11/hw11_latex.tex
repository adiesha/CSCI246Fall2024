\documentclass[12pt]{exam}
\setlength{\oddsidemargin}{0in}
\setlength{\evensidemargin}{0in}
\setlength{\textwidth}{6.8in}
\setlength{\parindent}{0in}
\setlength{\parskip}{\baselineskip}

\usepackage{graphicx} % Required for inserting images
\usepackage{amsmath,amsfonts,amssymb}
\usepackage{xcolor}
\usepackage{hyperref}
\usepackage{ dsfont }
\usepackage{tikz}
\title{CSCI-246 Discrete Structures HW 11}
\author{Instructor: Adiesha Liyanage}
\date{November 15 2024}

\begin{document}

\maketitle

\hrulefill
\\
\\
\textbf{Objective}
\begin{itemize}
    \item Understanding sum and product rules.
    \item Understanding probability and random variables.
    \item Understanding the problem solving process.
\end{itemize}

\textbf{Submission requirements}
\begin{itemize}
    \item \textbf{\textit{Type or clearly hand-write}} your solutions into a \textbf{\textit{PDF FORMAT.}} 
    \item \textbf{\textit{DO NOT UPLOAD images.}}
    \item \textbf{\textit{non-pdf or emailed solutions will not be graded.}}
    \item \textbf{If you take pictures of your handwritten homework, put it into pdf format.}
    \item \textbf{\textit{Start each problem on a new page.}}
    \item Follow the model that you have learned during the lectures for proofs.
    \item Do not wait until the last minute to submit the assignment.
    \item You can submit any number of times before the deadline. 
    \item If you are using latex, and you do not know how to type a symbol, use the following website. You can draw the symbol here and it will give you the latex code and the packages that you have to import. \url{https://detexify.kirelabs.org/classify.html}
    \item If you are using latex to write the answer, you can use overleaf to make your life easier. \textbf{Overleaf is a free, online platform that helps users create and publish scientific and technical documents using LaTeX, a markup-based document preparation system}
    \item If you do not understand a problem, ask questions during/after the lectures, or during office hours or via discord.
    \item Go to TA office hours and talk with them and ask for help.
    \item \textbf{\textit{Do not use generative AI to write answers.}} 
\end{itemize}

Homework 02 contains \textbf{3 questions}.

\section{Q1}
Suppose Inclusion-Exclusion for three sets $A,B,C$ is as following:
\[|A \cup B \cup C| = |A| + |B| + |C| - |A \cap B | - |A \cap C| - |B \cap C| + |A \cap B \cap C|\]

\begin{enumerate}
    \item Let $A = \{0,1,2,3,4\}, B = \{0,2,4,6\}, C = \{0,3,6\}$. Find the value of $A \cup B \cup C$ using inclusion-exclusion rule. \textbf{(You have to use the above rule)}.
    \item Consider the words \textbf{ONE, TWO, THREE, FOUR, FIVE, SIX, SEVEN,} and \textbf{EIGHT.} Let $E$ be the set of these words containing at least one \textbf{E}, let $T$ be the words containing a \textbf{T}, and let $R$ be the words containing an \textbf{R}. Then, calculate $|E \cap T \cap R|$.
\end{enumerate}

\section{Q2} 
In the United States, a text message can be sent either to a regular 10-digit phone number, or to a so-called short code which is a 5- or 6-digit number. Neither a phone number nor a short code can start with a 0 or a 1. How many different textable numbers are there in the United States?.

Hint: Use the product rule.


\section{Q3}
Sam flips a fair coin $100$ times. Let the outcome be the number of heads that he sees.


\begin{enumerate}
    \item What is the sample space? (represent the sample space using sets).
    \item For Sam's 100 flips, what is the $Pr[0]$ (What is the probability that you see 0 heads)?
    \item For Sam's 100 flips, what is $Pr[50]$?
    \item For Sam's 100 flips, what is $Pr[64]$?
\end{enumerate}


\section{Q4}
Suppose John flips a fair coin $n$ times. What is the probability of the event "There are strictly more heads than tails" if $n=2$?
\end{document}
