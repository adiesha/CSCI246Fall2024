\documentclass[12pt]{exam}
\setlength{\oddsidemargin}{0in}
\setlength{\evensidemargin}{0in}
\setlength{\textwidth}{6.8in}
\setlength{\parindent}{0in}
\setlength{\parskip}{\baselineskip}

\usepackage{graphicx} % Required for inserting images
\usepackage{amsmath,amsfonts,amssymb}
\usepackage{xcolor}
\usepackage{hyperref}
\usepackage{ dsfont }
\title{CSCI-246 Discrete Structures HW 5}
\author{Instructor: Adiesha Liyanage}
\date{September 22 2024}

\begin{document}

\maketitle

\hrulefill
\\
\\
\textbf{Objective}
\begin{itemize}
    \item Understanding predicate logic and quantifiers
    \item Mathematical definitions.
    \item How to approach solving a problem.
\end{itemize}

\textbf{Submission requirements}
\begin{itemize}
    \item \textbf{\textit{Type or clearly hand-write}} your solutions into a \textbf{\textit{PDF FORMAT.}} 
    \item \textbf{\textit{DO NOT UPLOAD images.}}
    \item \textbf{\textit{non-pdf or emailed solutions will not be graded.}}
    \item \textbf{If you take pictures of your handwritten homework, put it into pdf format.}
    \item \textbf{\textit{Start each problem on a new page.}}
    \item Follow the model that you have learned during the lectures for proofs.
    \item Do not wait until the last minute to submit the assignment.
    \item You can submit any number of times before the deadline. 
    \item If you are using latex, and you do not know how to type a symbol, use the following website. You can draw the symbol here and it will give you the latex code and the packages that you have to import. \url{https://detexify.kirelabs.org/classify.html}
    \item If you are using latex to write the answer, you can use overleaf to make your life easier. \textbf{Overleaf is a free, online platform that helps users create and publish scientific and technical documents using LaTeX, a markup-based document preparation system}
    \item If you do not understand a problem, ask questions during/after the lectures, or during office hours or via discord.
    \item Go to TA office hours and talk with them and ask for help.
    \item \textbf{\textit{Do not use generative AI to write answers.}} 
\end{itemize}

Homework 02 contains \textbf{3 questions}.

\section{Q1}
For the following claim, disprove this by proving a counterexample or provide a proof. Note that predicates $P,Q$ are defined on the set $S$. 

\[\forall x \in S : [ P(x) \vee Q(x) ] \iff [\forall x \in S : P(x)] \vee [\forall x \in S : Q(x)]\]

Hint 1: Note that $P, Q$ could be any predicate and $S$ could be any set. 

Hint 2: Try this claim with different predicates. Specially, try isEven(x) and IsOdd(x) predicates defined on the set of Integers ($\mathds{Z}$).

Hint 3: When you are trying to prove and if and only if statement, you need to prove both directions. For example, if you need to prove $p \iff q$, you need to prove $p \implies q$ and $q \implies p$. 


\section{Q2} 
Let $U$ be the set of all people. Let M(x) be the predicate where x is a student at MSU. Let F(X) be the predicate where x plays football. Let T(x) be the predicate where x plays Tennis. Let E(x) be the predicate where x earns money. Let P(x) be the predicate where x can punt. For each of the following, give an equivalent fully quantified expression. 

\begin{enumerate}
    \item Every football player who is a MSU student can play Tennis. 
    \item Every football player can punt but not every basketball player can punt. 
    \item Not everyone who plays Tennis earns money.
    \item At least one student at MSU can play football, basketball and earn money.
    \item All MSU students who can play basketball and tennis earn money.
\end{enumerate}

\end{document}
