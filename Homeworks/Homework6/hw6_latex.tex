\documentclass[12pt]{exam}
\setlength{\oddsidemargin}{0in}
\setlength{\evensidemargin}{0in}
\setlength{\textwidth}{6.8in}
\setlength{\parindent}{0in}
\setlength{\parskip}{\baselineskip}

\usepackage{graphicx} % Required for inserting images
\usepackage{amsmath,amsfonts,amssymb}
\usepackage{xcolor}
\usepackage{hyperref}
\usepackage{ dsfont }
\title{CSCI-246 Discrete Structures HW 6}
\author{Instructor: Adiesha Liyanage}
\date{September 28 2024}

\begin{document}

\maketitle

\hrulefill
\\
\\
\textbf{Objective}
\begin{itemize}
    \item Understanding functions
    \item Understanding Onto and One-To-One functions.
    \item Understanding how to prove a function is Onto and One-to-One.
\end{itemize}

\textbf{Submission requirements}
\begin{itemize}
    \item \textbf{\textit{Type or clearly hand-write}} your solutions into a \textbf{\textit{PDF FORMAT.}} 
    \item \textbf{\textit{DO NOT UPLOAD images.}}
    \item \textbf{\textit{non-pdf or emailed solutions will not be graded.}}
    \item \textbf{If you take pictures of your handwritten homework, put it into pdf format.}
    \item \textbf{\textit{Start each problem on a new page.}}
    \item Follow the model that you have learned during the lectures for proofs.
    \item Do not wait until the last minute to submit the assignment.
    \item You can submit any number of times before the deadline. 
    \item If you are using latex, and you do not know how to type a symbol, use the following website. You can draw the symbol here and it will give you the latex code and the packages that you have to import. \url{https://detexify.kirelabs.org/classify.html}
    \item If you are using latex to write the answer, you can use overleaf to make your life easier. \textbf{Overleaf is a free, online platform that helps users create and publish scientific and technical documents using LaTeX, a markup-based document preparation system}
    \item If you do not understand a problem, ask questions during/after the lectures, or during office hours or via discord.
    \item Go to TA office hours and talk with them and ask for help.
    \item \textbf{\textit{Do not use generative AI to write answers.}} 
\end{itemize}

Homework 02 contains \textbf{3 questions}.

\section{Q1}
Let $f(x) = \frac{7x}{5}$. Moreover let $2\mathds{Z}$ be the set of integers that are divisible by $c \in \mathds{Z}$. For example:

\[2\mathds{Z} = \{\ldots, -4, -2, 0, 2, 4, 6, \ldots\}\]
\[3\mathds{Z} = \{\ldots, -6, -3, 0, 3, 6, 9, \ldots\}\]

\begin{enumerate}
    \item Suppose $f$ is a mapping defined as $f : \mathds{Z} \rightarrow \mathds{Z}$. Show that $f$ is not a function.
    \item Suppose $f$ is defined as $f: 5\mathds{Z} \rightarrow \mathds{Z}$, then show that $f$ is a function.
    \item Suppose $f$ is defined as $f: 5\mathds{Z} \rightarrow \mathds{Z}$, then show that $f$ is \textbf{not a ONTO function.}
\end{enumerate}

Hint 1: Use the $3$ properties of a function that you learnt in the class. Show that one of the properties is violated by $f$.

Hint 2: If you want to show that $f$ is a function, then show that $f$ follows $3$ properties of a function.

Hint 3: If you want to show that a given function $f: A \rightarrow B$ is not a ONTO function, then what you need to show is that $\exists b \in B: [\forall a \in A: [f(a) \neq b]]$. In other words, there is at least one element in the codomain, where there is no element in the domain that is mapped to it. Basically, pick an element in $B$ that you can see there is no value that is mapped to it, and prove that there cannot be any element in the domain that can be mapped to it.


\section{Q2} 


\begin{enumerate}
    \item Define the $f : \{0, 1, 2, 3\} \rightarrow \{0, 1, 2, 3\}$ as $f(x) = x$. Is $f$ onto?
    \item Define the $f : \{0, 1, 2, 3\} \rightarrow \{0, 1, 2, 3\}$ as $f(x) = x^2~mod~4$. Is $f$ onto?
    \item Define the $f : \{0, 1, 2, 3\} \rightarrow \{0, 1, 2, 3\}$ defined as $f(x) = (x^2 - x)\;mod\;4$? Is $f$ onto?
    \item Define the $f(x) = x^2~mod~8$ as a function $f : \{0, 1, 2, 3\} \rightarrow \{0, 1, 2, 3 , 4 , 5 , 6 , 7\}$, is $f$ one-to-one?
    \item Define the $f(x) = x^3~mod~8$ as a function $f : \{0, 1, 2, 3\} \rightarrow \{0, 1, 2, 3,4, 5, 6, 7\}$, is $f$ one-to-one?
    \item Define $f(0) = 3$, $f(1) = 1$, $f(2) = 4$, and $f(3) = 1$. For this function $f : \{0, 1, 2, 3\} \rightarrow \{0, 1,2, 3, 4, 5, 6, 7\}$, is $f$ one-to-one?
\end{enumerate}


\section{Q3}

Let $A = \{1, 2, 3, 4\}$, $B= \{ 6,7,8\}$, $C=\{4,5,6,7\}$.
Give an example of a function that satisfy:
\begin{enumerate}
    \item An Onto function $f: A \rightarrow B$.
    \item An one-to-one function $g: A \rightarrow C$.
    \item A not Onto function $h: A \rightarrow C$.
    \item A not one-to-one function $p: A \rightarrow C$.
    \item A bijection (both onto and one-to-one) function $t: A \rightarrow C$. 
\end{enumerate}

Hint: A function does not need to have a neatly defined mathematical expression to qualify as a function. As long as it assigns each element in the domain to a unique value in codomain, it satisfies the definition of a function. One possibility of defining a function would be to create a table that maps every element in your domain to a unique element in the codomain.

\end{document}
