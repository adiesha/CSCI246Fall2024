\documentclass[12pt]{exam}
\setlength{\oddsidemargin}{0in}
\setlength{\evensidemargin}{0in}
\setlength{\textwidth}{6.8in}
\setlength{\parindent}{0in}
\setlength{\parskip}{\baselineskip}

\usepackage{graphicx} % Required for inserting images
\usepackage{amsmath,amsfonts,amssymb}
\usepackage{xcolor}
\usepackage{hyperref}
\usepackage{ dsfont }
\title{CSCI-246 Discrete Structures HW 4}
\author{Instructor: Adiesha Liyanage}
\date{September 15 2024}

\begin{document}

\maketitle

\hrulefill
\\
\\
\textbf{Objective}
\begin{itemize}
    \item Understanding proof by contradiction, proof by contradiction, implications, and contrapositive claim of an implication.
    \item Mathematical definitions.
    \item How to approach solving a problem.
\end{itemize}

\textbf{Submission requirements}
\begin{itemize}
    \item \textbf{\textit{Type or clearly hand-write}} your solutions into a \textbf{\textit{PDF FORMAT.}} 
    \item \textbf{\textit{DO NOT UPLOAD images.}}
    \item \textbf{\textit{non-pdf or emailed solutions will not be graded.}}
    \item \textbf{If you take pictures of your handwritten homework, put it into pdf format.}
    \item \textbf{\textit{Start each problem in a new page.}}
    \item Follow the model that you have learned during the lectures for proofs.
    \item Do not wait until the last minute to submit the assignment.
    \item You can submit any number of times before the deadline. 
    \item If you are using latex, and you do not know how to type a symbol, use the following website. You can draw the symbol here and it will give you the latex code and the packages that you have to import. \url{https://detexify.kirelabs.org/classify.html}
    \item If you are using latex to write the answer, you can use overleaf to make your life easier. \textbf{Overleaf is a free, online platform that helps users create and publish scientific and technical documents using LaTeX, a markup-based document preparation system}
    \item If you do not understand a problem, ask questions during/after the lectures, or during office hours or via discord.
    \item Go to TA office hours and talk with them and ask for help.
    \item \textbf{\textit{Do not use generative AI to write answers.}} 
\end{itemize}

Homework 02 contains \textbf{3 questions}.

\section{Q1}
Recall that for the sets $A$ and $B$, the claim $A \subseteq B$ is equivalent to the following statement: If $x \in A$, then $x \in B$. (If $x$ is an arbitrary element in the set $A$, then $x$ is an element of set $B$.)

\begin{enumerate}
    \item Write the converse of this implication statement.
    \item Write the inverse of this implication statement.
    \item Write the contrapositive of this implication statement.
\end{enumerate}

\section{Q2} 
Let $S,T$ and $W$ be sets such that $S \cap T \subseteq W$ and suppose that $t \in T$, then $t \in \overline{S - W}$

\begin{enumerate}
    \item Write this claim as an implication.
    \item Use the proof by contradiction technique to show that this claim is correct.
\end{enumerate}

Hint: First draw a Venn diagram for this relation. This will help you to understand what you have to prove for this claim. 


\section{Q3}
Consider the claim: Let $n \in \mathds{Z}^{\geq 0}$. If $2n^4 + n + 5$ is odd, then $n$ is even.

\begin{enumerate}
    \item Write the equivalent contrapositive claim.
    \item Use the proof by contrapositive technique to show the given claim is true.
\end{enumerate}



\end{document}
