\documentclass[12pt]{exam}
\setlength{\oddsidemargin}{0in}
\setlength{\evensidemargin}{0in}
\setlength{\textwidth}{6.8in}
\setlength{\parindent}{0in}
\setlength{\parskip}{\baselineskip}

\usepackage{graphicx} % Required for inserting images
\usepackage{amsmath,amsfonts,amssymb}
\usepackage{xcolor}
\usepackage{hyperref}
\usepackage{ dsfont }
\title{CSCI-246 Discrete Structures HW 7}
\author{Instructor: Adiesha Liyanage}
\date{October 14 2024}

\begin{document}

\maketitle

\hrulefill
\\
\\
\textbf{Objective}
\begin{itemize}
    \item Understanding relations.
    \item Understanding properties of a relation.
    \item Understanding the problem solving process.
\end{itemize}

\textbf{Submission requirements}
\begin{itemize}
    \item \textbf{\textit{Type or clearly hand-write}} your solutions into a \textbf{\textit{PDF FORMAT.}} 
    \item \textbf{\textit{DO NOT UPLOAD images.}}
    \item \textbf{\textit{non-pdf or emailed solutions will not be graded.}}
    \item \textbf{If you take pictures of your handwritten homework, put it into pdf format.}
    \item \textbf{\textit{Start each problem on a new page.}}
    \item Follow the model that you have learned during the lectures for proofs.
    \item Do not wait until the last minute to submit the assignment.
    \item You can submit any number of times before the deadline. 
    \item If you are using latex, and you do not know how to type a symbol, use the following website. You can draw the symbol here and it will give you the latex code and the packages that you have to import. \url{https://detexify.kirelabs.org/classify.html}
    \item If you are using latex to write the answer, you can use overleaf to make your life easier. \textbf{Overleaf is a free, online platform that helps users create and publish scientific and technical documents using LaTeX, a markup-based document preparation system}
    \item If you do not understand a problem, ask questions during/after the lectures, or during office hours or via discord.
    \item Go to TA office hours and talk with them and ask for help.
    \item \textbf{\textit{Do not use generative AI to write answers.}} 
\end{itemize}

Homework 02 contains \textbf{3 questions}.

\section{Q1}
Suppose you are given three relations $R_1, R_2, R_3$ defined on the set $S = \mathbb{P}(\{0,1,2,3\})$. Note that $R_1,R_2,R_3 \subseteq S \times S$. For each of these relations determine whether:

\begin{itemize}
    \item it is reflexive, irreflexive, or neither,
    \item it is symmetric, anti-symmetric, both, or neither,
    \item it is transitive or not, and
    \item all pairs of elements are comparable (that is, $\forall a \neq b \in S : (aRb \vee bRa))$.
\end{itemize}

\begin{enumerate}
    \item $R_1 \subseteq S \times S$ such that $(A, B) \in R_1$ if (i) $A$ and $B$ are nonempty and the largest element in $A$ equals the largest element in $B$, or (ii) if $A = B = \emptyset$.

    \item $R_2 \subseteq S \times S$ such that $(A, B) \in R_2$ if the sum of elements in $A$ is equal to the sum of elements in $B$. (Formally, $A\, R_2\,B \iff \sum_{x \in A} x =  \sum_{y \in B} y.$)

    \item $R_3 \subseteq S \times S$ such that $(A,B) \in R_3$ if $A \cap B \neq \emptyset$.
\end{enumerate}

Hint: Write down the set $S$ and $S\times S$. Then try to see which elements in $S \times S$ are related to each other under $R_1, R_2, R_3$ relations.

\section{Q2} 

In this problem, you will learn some new definitions. Note that we can define the size of a relation to be the number of elements in the corresponding set. For example, consider the set $A = \{1, 2, 3, 4\}$ and the relation $R = \{\langle 2, 4\rangle, \langle4, 3\rangle, \langle4, 4\rangle\}$. We say that $|R| = 3$.

We define the closure of a relation R with respect to some property as the smallest relation that includes everything from R but also has the property. To be precise:

\begin{itemize}
    \item The reflexive closure of a relation $R$ is the smallest possible $R' \supseteq R$ such that $R'$ is reflexive.
    \item The symmetric closure of a relation $R$ is the smallest possible $R' \supseteq R$ such that $R'$ is symmetric.
    \item The transitive closure of a relation $R$ is the smallest possible $R' \supseteq R$ such that $R'$ is transitive.
\end{itemize}

If R already has the property, then the closure is just $R$. For example, the reflexive closure of the $=$ relation on a set is just the $=$ relation, since $=$ is already reflexive.

For the set $A$ and relation $R \subseteq A \times A$ given above, give the following:

\begin{enumerate}
    \item The reflexive closure of R.
    \item The symmetric closure of R.
    \item The transitive closure of R.
\end{enumerate}
\end{document}
