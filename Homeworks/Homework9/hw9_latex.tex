\documentclass[12pt]{exam}
\setlength{\oddsidemargin}{0in}
\setlength{\evensidemargin}{0in}
\setlength{\textwidth}{6.8in}
\setlength{\parindent}{0in}
\setlength{\parskip}{\baselineskip}

\usepackage{graphicx} % Required for inserting images
\usepackage{amsmath,amsfonts,amssymb}
\usepackage{xcolor}
\usepackage{hyperref}
\usepackage{ dsfont }
\title{CSCI-246 Discrete Structures HW 9}
\author{Instructor: Adiesha Liyanage}
\date{October 28 2024}

\begin{document}

\maketitle

\hrulefill
\\
\\
\textbf{Objective}
\begin{itemize}
    \item Understanding mathematical induction.
    \item Understanding graph definitions.
    \item Understanding the problem solving process.
    \item Understanding summation notation.
\end{itemize}

\textbf{Submission requirements}
\begin{itemize}
    \item \textbf{\textit{Type or clearly hand-write}} your solutions into a \textbf{\textit{PDF FORMAT.}} 
    \item \textbf{\textit{DO NOT UPLOAD images.}}
    \item \textbf{\textit{non-pdf or emailed solutions will not be graded.}}
    \item \textbf{If you take pictures of your handwritten homework, put it into pdf format.}
    \item \textbf{\textit{Start each problem on a new page.}}
    \item Follow the model that you have learned during the lectures for proofs.
    \item Do not wait until the last minute to submit the assignment.
    \item You can submit any number of times before the deadline. 
    \item If you are using latex, and you do not know how to type a symbol, use the following website. You can draw the symbol here and it will give you the latex code and the packages that you have to import. \url{https://detexify.kirelabs.org/classify.html}
    \item If you are using latex to write the answer, you can use overleaf to make your life easier. \textbf{Overleaf is a free, online platform that helps users create and publish scientific and technical documents using LaTeX, a markup-based document preparation system}
    \item If you do not understand a problem, ask questions during/after the lectures, or during office hours or via discord.
    \item Go to TA office hours and talk with them and ask for help.
    \item \textbf{\textit{Do not use generative AI to write answers.}} 
\end{itemize}

Homework 02 contains \textbf{3 questions}.

\section{Q1}
Write the following sequences using summation notation. (This is an easy question, don't try to complicate this. I am not asking to give an expression, I am asking you to write the sequences using summation notation.)

\begin{itemize}
    \item $0 + 3 + 6 + 9 + \cdots + 3\cdot i + \cdots + 3\cdot n.$ (Sum of the first $n+1$ natural numbers.)
    \item $0^3 + 1^3 + 2^3 + \cdots + i^3 + \cdots + n^3.$ (Sum of the first $n+1$ cubes of natural numbers).
    \item $2^2 + 4^2 + 6^2 + \cdots + (2i)^2 + \cdots + (2\cdot n)^2.$ (Sum of the squares of first $n$ even numbers starting at 2.)
\end{itemize}

\section{Q2} 
Draw a graph with the nodes $V = \{1,2,\ldots, 9,10\}$, and edges between $x \in V$ and $y \in V$ if $x$ divides $y$. Does it make sense to use a directed or undirected graph? Is the graph that you have drawn simple?


\section{Q3}
In the class we looked at a property of an undirected graph $G=(V,E)$.
\[\sum_{v \in V} deg(v) = 2\cdot |E|\]
This property is known as \textbf{Handshaking Lemma}.

Here we will try to prove this using induction. The task for you is to fill in the missing pieces of this proof.

\textbf{Proof:} Here, I define the graph $G$ as a recursively defined structure. 

Undirected Graph $G$ can be defined as follows:
\begin{itemize}
    \item Base case: Start with an graph with no edges and \textbf{any number of isolated vertices} (each with degree zero).
    \item Recursive case: Add an edge between two vertices in the graph, updating the degrees of those two vertices.
\end{itemize}

Now that we have defined the undirected graph as a recursively defined structure, we can move to induction steps as follows:

For any natural number $m \geq 0$, let predicate $P(m)$ be true if any graph $G=(V,E)$ with $m$ edges has the following property: $\sum_{v \in V} deg(v) = 2\cdot m$-- false otherwise. We show that $\forall m \geq 0: P(m)$ using mathematical induction over $m$.

Base case: 

\textbf{You have to fill the base case and the proof of base case here.}

Inductive case:

We want to show that $\forall m \geq 1: P(m-1) \implies P(m)$. For the inductive hypothesis we assume that $P(m-1)$ is true; that is, we assume that \textbf{any} undirected graph $G=(V,E)$ with $m-1$ edges has $\sum_{v \in V} deg(v) = 2\cdot (m-1)$. Now let $H=(W, F)$ be any graph with $m$ edges. We want to prove that $P(m)$ holds; that is, $\sum_{v \in W} deg(v) = 2\cdot m$.

Let $\{u, w\}$ be any edge of $H$. Let us consider the graph $G$ constructed from $H$ by removing that one edge $\{u, w\}$ from $H$; that is, $G=(W, F \backslash \{\{u, w\}\})$ (basically $G$ is the graph that is obtained by removing the edge $\{u, w\}$). Note that we keep the same set of vertices; we only remove a single edge. For any vertex $v\in W$, let $deg_G(v)$ denote the degree of $v$ in graph $G$, and $deg_H(v)$ denote the degree in graph $H$, since these could be different once you remove the edge $\{u,w\}$.

\textbf{Fill the rest of the proof of the inductive case here.}

Since we showed that $P(0)$, and $\forall m \geq 1: P(m-1) \implies P(m)$, we can conclude that $\forall m \geq 0: P(m)$ using the principle of mathematical induction.


\end{document}
