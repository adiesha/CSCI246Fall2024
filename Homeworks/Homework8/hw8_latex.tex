\documentclass[12pt]{exam}
\setlength{\oddsidemargin}{0in}
\setlength{\evensidemargin}{0in}
\setlength{\textwidth}{6.8in}
\setlength{\parindent}{0in}
\setlength{\parskip}{\baselineskip}

\usepackage{graphicx} % Required for inserting images
\usepackage{amsmath,amsfonts,amssymb}
\usepackage{xcolor}
\usepackage{hyperref}
\usepackage{ dsfont }
\title{CSCI-246 Discrete Structures HW 8}
\author{Instructor: Adiesha Liyanage}
\date{October 18 2024}

\begin{document}

\maketitle

\hrulefill
\\
\\
\textbf{Objective}
\begin{itemize}
    \item Understanding partial orders.
    \item Understanding steps of a induction.
    \item Understanding the problem solving process.
\end{itemize}

\textbf{Submission requirements}
\begin{itemize}
    \item \textbf{\textit{Type or clearly hand-write}} your solutions into a \textbf{\textit{PDF FORMAT.}} 
    \item \textbf{\textit{DO NOT UPLOAD images.}}
    \item \textbf{\textit{non-pdf or emailed solutions will not be graded.}}
    \item \textbf{If you take pictures of your handwritten homework, put it into pdf format.}
    \item \textbf{\textit{Start each problem on a new page.}}
    \item Follow the model that you have learned during the lectures for proofs.
    \item Do not wait until the last minute to submit the assignment.
    \item You can submit any number of times before the deadline. 
    \item If you are using latex, and you do not know how to type a symbol, use the following website. You can draw the symbol here and it will give you the latex code and the packages that you have to import. \url{https://detexify.kirelabs.org/classify.html}
    \item If you are using latex to write the answer, you can use overleaf to make your life easier. \textbf{Overleaf is a free, online platform that helps users create and publish scientific and technical documents using LaTeX, a markup-based document preparation system}
    \item If you do not understand a problem, ask questions during/after the lectures, or during office hours or via discord.
    \item Go to TA office hours and talk with them and ask for help.
    \item \textbf{\textit{Do not use generative AI to write answers.}} 
\end{itemize}

Homework 02 contains \textbf{3 questions}.

\section{Q1}
List a partial order, strict partial order and a equivalence relation that you can create from the following set $A=\{0,1,2,3\}$.

\section{Q2} 
For a given integer $n \geq 0$, consider the sum of first $n$ cubes: $0^3 + 1^3 + 2^3 + \cdots + n^3$. You can write the sum of first $n$ cubes as $\sum_{i=0}^{n}i^3$. We can hypothesize that this sum of cubes of first $n$ natural numbers are $\sum_{i=0}^{n}i^3 = (\frac{n\cdot(n+1)}{2})^2$. 

\begin{enumerate}
    \item Show that this formula works for $0 \leq n \leq 3$. (This is very simple try to manually check the sum of cubes for $0 \leq n \leq 3$ and check whether you get the same value from the formula.)
    \item Use mathematical induction to prove that this formula works for any integer $n \geq 0$.
    
    Hint: Try to model this problem into the induction framework that we learnt during the class. 
    \begin{itemize}
        \item define the predicate $P(n)$.
        \item State the variable that you are performing the induction over.
        \item State the base case.
        \item prove the base case.
        \item state the inductive case.
        \item prove the inductive case.
            \begin{itemize}
                \item assume the inductive hypothesis $P(n-1)$
                \item start with the \textbf{LHS} of the $P(n)$ and manipulate it to get the \textbf{RHS} (or vice versa.) \textbf{Do not start with }\textbf{LHS = RHS}.
                \item fina a way to get the $P(n-1)$ in your algebra somewhere, so you can apply the inductive hypothesis.
                \item correctly apply the inductive hypothesis.
                \item clearly say that you have applied the inductive hypothesis.
                \item Then derive the \textbf{RHS} of the $P(n)$ (if you starrt with the \textbf{LHS}) or \textbf{LHS} (if you start with the \textbf{RHS}).
            \end{itemize}
        \item Finish the proof by tying everything together.
    \end{itemize}
\end{enumerate}


\section{Q3}
Suppose you want to calculate the sum of first $n$ odd numbers. For example, the sum of first $3$ odd numbers would be $1 + 3 + 5 = 9$. We can label this sum as $\sum_{i=1}^{n} (2i - 1)$. We can hypothesize that this sum is equal to $n^2$.

\begin{enumerate}
    \item Show that this formula works for $1 \leq n \leq 3$. (This is simple, manually check whether the \textbf{LHS} and \textbf{RHS} of the claim is equal).
    \item Use mathematical induction to prove that this formula is correct for $n \geq 1$.

    Hint: Try to model this problem into the induction framework that we learnt during the class. 
    \begin{itemize}
        \item define the predicate $P(n)$.
        \item State the variable that you are performing the induction over.
        \item State the base case.
        \item prove the base case.
        \item state the inductive case.
        \item prove the inductive case.
            \begin{itemize}
                \item assume the inductive hypothesis $P(n-1)$
                \item start with the \textbf{LHS} of the $P(n)$ and manipulate it to get the \textbf{RHS} (or vice versa.) \textbf{Do not start with }\textbf{LHS = RHS}.
                \item fina a way to get the $P(n-1)$ in your algebra somewhere, so you can apply the inductive hypothesis.
                \item correctly apply the inductive hypothesis.
                \item clearly say that you have applied the inductive hypothesis.
                \item Then derive the \textbf{RHS} of the $P(n)$ (if you starrt with the \textbf{LHS}) or \textbf{LHS} (if you start with the \textbf{RHS}).
            \end{itemize}
        \item Finish the proof by tying everything together.
    \end{itemize}
\end{enumerate}

\end{document}
